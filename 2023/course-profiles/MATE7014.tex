\hypertarget{MATE7014}{\section{MATE7014 - Advanced Materials Characterization}}

\large
\textcolor{turbo_purple}{\href{https://my.uq.edu.au/programs-courses/course.html?course_code=MATE7014}{Official Page}} \\
Rating: \cstar\cstar\cstar\cstar\ostar

\normalsize
\subsection*{Description}
Materials Characterization provides unique tools for understanding the materials and their demonstrated properties.
Materials Characterization techniques, such as x-ray diffraction, scanning electron microscopy, and transmission electron microscopy, allow detailed structural, chemical, and morphological characteristics of materials to be determined, which has become essential tools for materials research and their productions.
By corelating the determined structural and chemical characteristics of a material with its fabrication/processing, the formation mechanism of the material can be clarified.
This is vital for developing new material systems, and for identifying problems in the production lines.
On the other hand, the correlation of the determined structural and chemical characteristics of a material with its demonstrated properties allows the material's structure-property link to be built, which is critically important for understanding the origin of the properties.
For this reason, demand for learning various materials characterization techniques have increased sharply in the recent decades.

\subsection*{Review}
review here
